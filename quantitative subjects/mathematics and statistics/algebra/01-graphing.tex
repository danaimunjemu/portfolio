\documentclass{article}
\title{01 Graphing}
\author{Danai Munjemu}
\date{\today}

\begin{document}

\maketitle %The \maketitle command generates a title

\section{Introduction}
When a graph is plotted, there is usually a way to translate the relationship between the $x$ axis and the $y$ axis. For example, if $x$ represents time and $y$ represents distance, then:
$\frac{x}{y}$ represents \textit{velocity}.

\section{Cartersian Coordinate System}
The (textit{reference}) point where the $x$ axis and $y$ axis meet is called the textit{origin}.

Points on a graph are represented as textit{ordered pairs} in the form ($x$, $y$). This will show the location of the point on the graph. This means that all locations on the graph will have a name - textbf{the ordered pair}. There is exactly one ordered pair that is going to name that location.
The sign (+ or -) gives direction to the name.

Every single point on the plane has exactly one name and every ordered pair has exactly on location.

This system - textit{the cartesian coordinate system} was named after Rene Descartes - a philosopher and scientist from France.

\section{Intercepts}
A curve intercepts the $x$ or $y$ axis where it meets it. A textit{y-intercept} is the place where a line or curve crosses, or touches, the y-axis - the vertical.  An textit{x-intercept} is the place where a line or curve crosses, or touches, the x-axis - the horizontal.

\section{Graphing an Equation}
Given an equation:
$$y = 2x + 3$$
You can represent this on a graph. There will be infinite pairs to graph. So you can decide on pairs to represent:

\begin{tabular}{r|r}
  $x$ & $y$\\
  \hline
  0 & 3\\
  2 & 7\\
  -2 & -1\\
  3 & 9\\
\end{tabular}
  
The point $(0,3)$ is the y-intercept
The x-intercept is $(-1.5, 0)$

There are infinite points along the line.
It is important to take note of the terms:
\begin{itemize}
  \item ordered pair
  \item equation
\end{itemize}


Take another example:

$$y = -4$$
The y-intercept is $(0, -4)$
There is no x-intercept

It is always a good idea - when you get an equation - to look for the intercepts.
Remember, the graph can cross an axis more than once - there can be more than 1 intercept on either axis.

A mixed number, also known as a mixed fraction, is a number that is made up of a whole number and a fraction.
An example is:
$2 \frac{3}{4}$

An important point is to always analyse what is shown on the graph. What is shown by the x-coordinate and the y-coordinate.

NOT ALL RELATIONSHIPS ARE LINEAR.

\end{document}
