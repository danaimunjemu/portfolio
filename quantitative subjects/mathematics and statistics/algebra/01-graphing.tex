\documentclass{article}
\title{01 Graphing}
\author{Danai Munjemu}
\date{\today}

\begin{document}

\maketitle %The \maketitle command generates a title

\section{Introduction}
When a graph is plotted, there is usually a way to translate the relationship between the $x$ axis and the $y$ axis. For example, if $x$ represents time and $y$ represents distance, then:
$\frac{x}{y}$ represents \textit{velocity}.

\section{Cartersian Coordinate System}
The (textit{reference}) point where the $x$ axis and $y$ axis meet is called the textit{origin}.

Points on a graph are represented as textit{ordered pairs} in the form ($x$, $y$). This will show the location of the point on the graph. This means that all locations on the graph will have a name - textbf{the ordered pair}. There is exactly one ordered pair that is going to name that location.
The sign (+ or -) gives direction to the name.

Every single point on the plane has exactly one name and every ordered pair has exactly on location.

This system - textit{the cartesian coordinate system} was named after Rene Descartes - a philosopher and scientist from France.


\end{document}
